\documentclass[10pt]{article}
\usepackage{amsfonts,amssymb,amsmath,amsthm,ifthen}

\setlength{\textwidth}{175mm}
\setlength{\textheight}{255mm}
\setlength{\topmargin}{-20mm}
\setlength{\oddsidemargin}{-7mm}
\setlength{\evensidemargin}{-7mm}
\pagestyle{empty}

% Packages
\usepackage{graphicx}
\graphicspath{{Figures/}}
\newcommand{\figscale}{0.4}
\usepackage{comment} %Comment environment
\usepackage{float}
\usepackage{multirow} %Tables
\usepackage{hyperref}

%For C++ Code
\usepackage{listings}
\usepackage{xcolor}
\lstset { % 
    language=C++,
    backgroundcolor=\color{black!5}, % set backgroundcolor
    basicstyle=\footnotesize,% basic font setting
    showstringspaces=false,% no squat-u as part of strings
}

% Math Layouts
\newtheorem{theorem}{Theorem}[section]
\newtheorem{lemma}[theorem]{Lemma}
\newtheorem{claim}[theorem]{Claim}
\newtheorem{proposition}[theorem]{Proposition}
\newtheorem{corollary}[theorem]{Corollary}
\newtheorem{remark}[theorem]{Remark}
\newtheorem{definition}[theorem]{Definition}
\newtheorem{example}[theorem]{Example}

% Additional Commands
\newcommand{\tsim}{$\sim$}
\newcommand{\tlangle}{$<$}
\newcommand{\trangle}{$>$}

% Title
\title{Workflow Notes}
\author{Hwan Goh}
\date{17th May 2020}


%---------------------------------------------------------------
%                         Document Body 
%---------------------------------------------------------------
\begin{document}
\maketitle

%===============================================================
\section{Vim Tips and Tricks}
%===============================================================
Some good references are \cite{chang2018vim,toomey2015mastering}. A more
advanced tutorial that I haven't watched yet is \cite{chang2020vim}. A document
on effective text editing written by the creator of Vim can be found at
\cite{moolenaar2000seven}.
\begin{itemize}
    \item In Linux based systems, put `setxkbmap -option "caps:escape' into
        ~/.bashrc to map the caps lock key to escape.
    \item To set Vim as your default text editor, use `sudo update-alternatives
        --config editor'.  \item Suppose the cursor is in the middle of a word.
        Whilst `cw' and `dw' will change/delete until the end of the word, `ciw'
        and 'diw' to change or delete the whole word, thereby not requiring you
        to move the cursor to the beginning of the word first.
    \item To reformat a paragraph, use `gq'. Reformatting includes enforcing the
        wrap limit which can be set in your vimrc with `set tw=\tlangle
        number\trangle'. Other options can be se such as indentation. Note that
        this command requires a minimum of two lines, so you'll need to at least
        use `gqj' and at most use `gjG' for the rest of the document under the
        line currently under the cursor.
    \item Use `m\tlangle key\trangle to set a mark to \tlangle key\trangle then
        \tlangle key\trangle to jump to it.  Note that if you set it to the
        capitalized version of \tlangle key\trangle then the jumping can occur
        between buffers. To see your list of marks, use `:mark'.
    \item Use `"\tlangle key\trangle' then an action like `y' or `d' to store
        text in the \textit{register} \tlangle key\trangle. Then use `"\tlangle
        key\trangle p' to paste it. Note that using `d' will automatically store
        it in the register at register x and `p' will automatically paste
        whatever is in register `x'.  Use `:reg' to see the register list.
    \item Use `f\tlangle key\trangle' to forward search for \tlangle
        key\trangle. To backwards search, use `F\tlangle key\trangle'. To repeat
        the search, use `;'. 
    \item The `t' stands for `til'. For example, `dt=' will delete up to and not
        including an = sign.  \item The `*' can be used to search for the word
        under cursor.
    \item `J' is used to join the line below to the line currently under the
        cursor. This is useful for reformatting improperly wrapped lines but in
        general `gq' is more useful for this purpose.
    \item `zt' or `z<CR>' will put the line under the cursor to the to of the
        window. `z.' will put the line to the center of the window and `z-' will
        put it to the bottom of the window.
    \item `z=' will give spelling suggestions for the word under the cursor.
    \item `[s' and `]s' to cycle backwards and forwards through misspelled words.
    \item Use `g\$' to go to the end of an unwrapped line.
    \item If you type a long line of text in insert mode, this counts as one
        action. Therefore, if you press undo, the whole line will be deleted. To
        break the undo chain, use \tlangle c-g\trangle u while in insert mode.
        Alternatively, you could map space to \tlangle c-g\trangle u so that the
        undo chain is broken whenever a space is added in insert mode:\\
        inoremap \tlangle Space\trangle \tlangle Space\trangle \tlangle
        c-g\trangle u
    \item Use the accent `\tlangle shift+6\trangle' in insert mode to move to
        the first non-white space character in the line.
    \item Use `\%' while your cursor is over a parenthesis, square or curly
        bracket to move to the corresponding open/closing parenthesis or
        bracket.
    \item Use `:ls' to see the list of buffers then `:b' and the number to
        select one. A useful mapping for this process is:\\ nnoremap \tlangle
        leader\trangle b:ls\tlangle cr\trangle:b\tlangle space\trangle
    \item Use `\tlangle number\trangle + <c-6>' to switch to numbered buffer.
\end{itemize}

%===============================================================
\section{Plugin Management}
%===============================================================
The first part of this section follows \cite{neil2010synchronizing}. First, we
work towards turning ~/.vim into a git repository:
\begin{enumerate}
    \item Move \tsim/.vimrc into \tsim/vim.
    \item When vim boots, it's still going to look for .vimrc in the home
        directory. To ensure that it looks for vimrc in the \tsim/.vim
        directory, we can create a symbolic link to that file using `ln -s
        \tsim/.vim/vimrc \tsim/.vimrc'.  \item Make \tsim/.vim into a git
        repository.
\end{enumerate}
An issue now is if you install a plugin that itself is a git repository, you
lose the version-control capabilities of that plugin. To circumvent this issue,
we use a plugin manager; in this case \textit{Pathogen}.\\

%---------------------------------------------------------------
\subsection{Pathogen} \label{SecPathogen}
%---------------------------------------------------------------   
The pathogen plugin makes it possible to cleanly install plugins as a bundle.
Rather than having to place all of your plugins side by side in the same
directory, you can keep all of the files for each individual plugin together in
one directory (see video from first link for example). This makes installation
more straightforward, and also simplifies the tasks of upgrading and even
removing a plugin if you decide you no longer need it since they are carefully
segregated from each other. For a good tutorial on Pathogen, see
\cite{lafourcade2014how}.\\
 
Following the readme on the repo at \cite{pope2009pathogen}, to install Pathogen
do the following: 
\begin{enumerate}
    \item Run in terminal:\\
        mkdir -p \tsim/.vim/autoload \tsim/.vim/bundle \&\& \ \\ curl -LSso
        \tsim/.vim/autoload/pathogen.vim https://tpo.pe/pathogen.vim
    \item Add the following to your vimrc:
        execute pathogen\#infect()
\end{enumerate}
Now any plugins you wish to install can be extracted to a subdirectory under
~/.vim/bundle, and they will be added to the 'runtimepath'. For example, to
install "sensible.vim", simply run: "cd ~/.vim/bundle \&\& \ git clone
https://github.com/tpope/vim-sensible.git".

%---------------------------------------------------------------
\subsection{Submodules: Installing Git Repositories Within Git Repositories}\label{SecSubmodules}
%---------------------------------------------------------------   
This section follows \cite{neil2010synchronizing}. Now that we can install
plugins via Pathogen, let's see how we preserve the version control capabilities
of our plugins. As a worked example, let us install the \textit{Vimtex} plugin:
\begin{enumerate}
    \item cd \tsim/.vim
    \item Now to clone a git repository into the bundle directory, use:\\ git
        submodule add https://github.com/lervag/vimtex.git bundle/vimtex
\end{enumerate}
Now, to upgrade this plugin, use:
\begin{enumerate}
    \item cd \tsim/.vim/bundle/vimtex
    \item git pull origin master
\end{enumerate}
To upgrade ALL of your plugins, use:
\begin{enumerate}
    \item cd \tsim/.vim
    \item git submodule foreach git pull origin master
\end{enumerate}
%---------------------------------------------------------------
\subsection{Importing Your Vim Configuration and Plugins To a New Machines}
%---------------------------------------------------------------   
One of the main benefits of version controlling your Vim configuration and
plugins is the ease of which they can be imported into a new machine. To do so,
use the following:
\begin{enumerate}
    \item cd \tsim 
    \item git clone \tlangle git repo url\trangle \tsim/.vim
    \item ln -s \tsim/.vim/vimrc \tsim/.vimrc
    \item cd \tsim/.vim
    \item git submodule init
    \item git submodule update
\end{enumerate}
%---------------------------------------------------------------
\subsection{Vim-plug}
%---------------------------------------------------------------   
Whilst Pathogen is the most basic plugin manager, there are limitations when
porting your setup to a new machine:
\begin{enumerate}
    \item When you run git submodule update, you'll pull the latest versions of
        the plugins from their respective repositories. So unless you've noted
        down somewhere all the commit IDs for your favourite version of each
        plugin, your overall plugin collection cannot be preserved when porting
        to a new machine.
    \item Suppose one of the plugins is no longer being maintained by the owner
        and suppose also that you've made your own changes to the plugin. If you
        were to pull your configuration on a new machine, you will pull the
        latest version of the plugin; that is your changes will not be ported.
        Further, on your own repository for your configuration, the directories
        containing the plugins will be treated as repositories. Unless you
        manually install the plugins, there is no way to preserve your changes
        onto Github.  
\end{enumerate}
The plugin manager Vim-plug \cite{junegunn2014vimplug} has a solution to the
first problem. This plugin manager is extremely simple to use:
\begin{enumerate}
    \item Run in terminal:\\
        curl -fLo \tsim/.vim/autoload/plug.vim --create-dirs \
        https://raw.githubusercontent.com/junegunn/vim-plug/master/plug.vim
    \item In your vimrc, add the line:\\
        call plug\#begin('\tsim/.vim/plugged')
    \item To include a plugin you wish to install under the line above, in your
        vimrc add the following line:\\ 
        Plug 'https://github.com/lervag/vimtex.git'
    \item Under `Plug \tlangle url \trangle' of your last plugin, in your vimrc
        add the following line:\\ 
        call plug\#end()
    \item Reload your vimrc and use the command `:PlugInstall' while in vim. Now
        all your plugins are installed.  
\end{enumerate}
Now to preserve the versions of your plugin:
\begin{enumerate}
    \item In Vim, run the command `:PlugSnapshot! \tlangle filename \trangle.vim'.
    \item To restore the state of your plugins, in vim run the command:\\
        :source \tsim/.vim/\tlangle filename \trangle.vim\\
        or, in terminal:\\
        vim -S \tsim/.vim/\tlangle filename \trangle.vim
\end{enumerate}
See the bradagy's answer in the reddit post \cite{bradagy2018remember}. To port
your configuration to a new machine:
\begin{enumerate}
    \item cd \tsim
    \item git clone \tlangle git repo url \trangle \tsim/.vim
    \item cd \tsim/.vim/vimrc
    \item :PlugInstall
    \item :source \tsim/.vim/\tlangle filename \trangle.vim
\end{enumerate}
To uninstall plugins:
\begin{enumerate}
    \item Delete `Plug \tlangle url \trangle' line from your vimrc
    \item :PlugClean
\end{enumerate}
%===============================================================
\section{Native Plugin Management} \label{SecNativePluginManagement}
%===============================================================
Vim-plug provides a solution to the first of the two issues mentioned in the
previous section, but not the second. To address the second issue, we can use
Vim's native plugin management system to install a plugin and delete the .git
repository. That way, we can track the files and any changes in our own git
repository. This section follows \cite{manasthakur2020managing}. For a more
detailed explanation, see \cite{ryder2018attack}.\\

The package feature of Vim 8 follows a pathogen-like model and adds the plugins
found inside a custom-path \tsim/.vim/pack/ to Vim's runtime path. You can check
the version of Vim installed using `vim --version'. To install using this native
feature, we use the following steps:
\begin{enumerate}
    \item mkdir -p \tsim/.vim/pack/plugins/start/
    \item git clone \tlangle url\trangle
    \item :helptags \tsim/.vim/pack/plugins/start/\tlangle plugin name\trangle
\end{enumerate}
and that's it! To remove a plugin, simply remove its directory: rm -r
\tsim/.vim/pack/plugins/start/foo.

%===============================================================
\section{Vimtex}
%===============================================================
%---------------------------------------------------------------
\subsection{Installing and Running Vimtex}
%---------------------------------------------------------------   
Assuming you are using the Pathogen plugin manager and version controlling your
~/.vim directory, follow the steps used in Section \ref{SecSubmodules}. Then use
`:Helptags' which is Pathogen's method for generating help tags. With this, we
can use `:h vimtex' to see the manual for Vimtex. To confirm that the plugin
works, type ":VimtexInfo" to see a summary of the tex file.
%---------------------------------------------------------------
\subsection{Compiling a Tex File}
%---------------------------------------------------------------   
The following commands are useful:
\begin{itemize}
    \item :VimtexCompile \# this is a continuous compiler meaning that everytime
        you save with ":w" it will automatically compile
    \item :VimtexStop \# this stops the continuous compiler
    \item :VimtexCompileSS \# this is a single shot compiler. Note that you have
        to save your file first \item :VimtexClean \# Cleans auxiliary files
        generated in compilation process 
\end{itemize}
I set the following mappings in my vimrc:
\begin{lstlisting}
autocmd FileType tex nnoremap <F5> :VimtexView<Enter>
autocmd FileType tex inoremap <F5> <Esc> :VimtexView<Enter>
autocmd FileType tex nnoremap <F6> :w! <bar> :VimtexCompileSS<Enter>
autocmd FileType tex inoremap <F6> <Esc> :w! <bar> :VimtexCompileSS<Enter>
\end{lstlisting}

%---------------------------------------------------------------
\subsection{Forward and Backwards Searching with Synctex}
%---------------------------------------------------------------   
This section follows \cite{gunther2014vimtex}. For this, you will need a Vimtex
server which allows you to do forward and backward to navigate between
corresponding sections of the tex file and the pdf. You will also need the pdf
viewer \textit{Zathura}. To install, simply use `sudo apt-get install zathura'.
Then, in your vimrc, add the following line `let g:vimtex\_view\_method =
'zathura''.\\

Following \cite{lerner2004enable}, to install just use `sudo apt-get install
vim-gnome'. Finally, to edit a tex file with navigation capabilities:
\begin{enumerate}
    \item In the directory containing the tex file, use:\\
        vim --servername \tlangle servername\trangle \tlangle texfile\trangle.tex
    \item While in tex file, to jump to the text on the pdf corresponding to the
        line under your cursor, use:\\ 
        \tlangle leader\trangle lv
    \item Move your mouse cursor over some text on the pdf. Then, to jump to
        corresponding text on the tex file, use:\\ 
        Ctrl + left-click
\end{enumerate}
Note that \tlangle leader\trangle lv forward search works without the Vim
server; it's the backwards search that requires the Vim server.

%---------------------------------------------------------------
\subsection{Other Useful References}
%---------------------------------------------------------------   
\begin{itemize}
    \item Nice compilation of Vimtex commands \cite{gunther2014vimtex}
    \item Quick guide on the basics of Vimtex \cite{jdhao2019complete}
    \item For a comparison with other Vim LaTeX plugins \cite{lervag2015vim}
    \item Note-taking using Vim and LaTeX for math lectures \cite{castel2019how}
    \item Vim and LaTeX on MacOS \cite{dyke2020getting}
    \item Some useful mappings such as the teleportation trick \cite{smith2016my, smith2017start}. 
\end{itemize}

%===============================================================
\section{IPython}
%===============================================================
The following mapping may prove useful for starting an IPython terminal:\\ 
\begin{lstlisting}
nnoremap <leader>P :botright vertical terminal ipython --no-autoindent<CR><C-w><left>
\end{lstlisting}

%---------------------------------------------------------------
\subsection{sendtowindow Plugin}
%---------------------------------------------------------------   
When coding in Python with a Vim terminal on the side running IPython, I use the
\textit{sendtowindow} plugin \cite{KKPMW2016send} to send lines of code to the
terminal. See \cite{KKPMW2019send} for a Reddit post from the author discussing
the plugin. It's a very simple plugin that allows the use of vim motions to move
lines of text to terminals left, right, above or below the Vim window. I use the
following maps:\\
\begin{lstlisting}
let g:sendtowindow_use_defaults=0
nmap ,sr <Plug> SendRight
xmap ,srv <Plug> SendRightV
nmap ,sl <Plug> SendLeft
xmap ,slv <Plug> SendLeftV
nmap ,su <Plug> SendUp
xmap ,suv <Plug> SendUpV
nmap ,sd <Plug> SendDown
xmap ,sdv <Plug> SendDownV
\end{lstlisting}
Note that these mappings must not be noremaps (no recursive mapping). For an
explanation on why this is the case, please see \cite{justrajdeep2018please}.
Any one of these mappings will specify to either send some text in normal mode
or in visual mode. When in normal mode, to specify which text to send, use the
usual Vim movements. For example, `,sr\$' will send from the cursor to the end
of the line to the window on the right. However, if you are in the middle of a
line, it may become tedious having to first go to the beginning of the line and
having to type `\$'. Therefore, we can define line objects using the following
maps:
\begin{lstlisting}
onoremap <silent> <expr> - v:count==0 ? ":<c-u>normal! 0V$h<cr>" : ":<c-u>normal! V" . (v:count) . "jk<cr>"
vnoremap <silent> <expr> - v:count==0 ? ":<c-u>normal! 0V$h<cr>" : ":<c-u>normal! V" . (v:count) . "jk<cr>"
onoremap <silent> <expr> i- v:count==0 ? ":<c-u>normal! ^vg_<cr>" : ":<c-u>normal! ^v" . (v:count) . "jkg_<cr>"
vnoremap <silent> <expr> i- v:count==0 ? ":<c-u>normal! ^vg_<cr>" : ":<c-u>normal! ^v" . (v:count) . "jkg_h<cr>"
\end{lstlisting}
With '\_' alone the indentation is left intact and 'i\_' is without indentation.
So, for example, `,sr\_' will send the line under the cursor to the window on
the right with indentation and `,sr8\_' will send four lines under the cursor to
the window on the right. Note that this is important for Python as if you try
send a for loop with `i\_' and leave out the indentation, then it may not run.\\

Finally, using these mappings, we can create a mapping that sends clear and
\%reset commands to the IPython terminal. I am aware that these mappings are as
hacky as the day is long, but they work:
\begin{lstlisting}
" Send to right window (I only really ever use the right window)
nmap ,s <Plug>SendRight
xmap ,s <Plug>SendRightV

" Insert current file name
inoremap ;F <C-R>=expand("%:t")<CR>

" Clear, Reset, Run Variable, Run Marked Section and Run Code
nmap ,sC mqA<CR>clear<Esc>V,suu`q 
nmap ,sD mqA<CR>%reset<Esc>V,suiy<Esc>v,suu`q
nmap ,sV mqviw,s`q
nmap ,sM mq'xV'z,s`q
nmap ,sR mqA<CR>run ;F<Esc>V,suuu`q
\end{lstlisting}
Be sure to use recursive mappings which will allow us to reuse the `,s' mapping
from before. Notice also that `mq' sets a mark at your current cursor location
and `q returns to that location after executing the command. Also, when using
marked sections, the plugin vim-signature \cite{kshenoy2015signature} may come
in handy for manipulating the marks.

%---------------------------------------------------------------
\subsection{With tmux}
%---------------------------------------------------------------   
There are many plugins that allow interaction between Vim and tmux. For example,
there is Vim-Slime \cite{jpalardy2012slime} and Vimux \cite{benmills2009vimux}.
The one I use is Slimux \cite{esamattis2015slimux}. There is a blog post about
it here \cite{suuronen2012slimux}. As discussed in the blog post, Slimux differs
from Vimux in that it is more disjoint from tmux. In particular, Vimux will
create a pane on which you must run your commands whereas slimux allows you to
select the pane you wish to use. Also, unlike Vim-Slime, you do not need to
manually type in the pane to select it; in Slimux an interactive prompt is
given. It's also worth noting that Vim-Slime supports many terminal multiplexers
such as GNU Screen, kitty and Vim's native terminal.\\

However, there is an issue with sending commands to IPython where the
indentation is constantly carried over \cite{kmARC2015indentationerror}. There
are three proposed fixes
\cite{lotabout2017remove,karadaharu2016add,zcesur2018fix}. The first modifies
python.vim, the second modifies both slimux.vim and python.vim by implementing
IPython's cpaste function and the third modifies a single line of slimux.vim.
Since the repo owner hasn't pulled any of these fixes, I opted for the third fix
which requires changing line 328 of slimux.vim to: 
\begin{lstlisting} let
b:code_packet["text"] = "\e[200~" . a:text . "\e[201~\r\r.  
\end{lstlisting}
Note that this repository has not been updated in years and so all pull requests
have not been fulfilled. Therefore, if you wish to track the fixes you've made,
you may want to install the plugin using the method disussed in Section
\ref{SecNativePluginManagement}.\\

I use the following mappings:
\begin{lstlisting}
nmap ,- :SlimuxREPLSendLine<CR>
vmap ,- :SlimuxREPLSendSelection<CR>
nmap ,T :SlimuxShellRun 
nmap ,l :SlimuxShellLast<CR>

" Clear, Reset, Run Variable, Run Marked Section and Run Code
nmap ,C :SlimuxShellRun clear<CR>
nmap ,D :SlimuxShellRun %reset<CR>:SlimuxShellRun y<CR>
nmap ,V mqviw,-`q
nmap ,M mq'xV'z,-`q
nmap ,R A<CR>run ;F<Esc>V,-uuu
\end{lstlisting}
where SlimexShellRun awaits a command to send to shell and SlimuxShellLast runs
the last shell command. 

%---------------------------------------------------------------
\subsection{Other Useful References}
%---------------------------------------------------------------   
\begin{itemize}
    \item A good Reddit thread on workflow with Vim and IPython can be found at
        \cite{abdeljalil732020anyone}. Note that the majority of the responses
        indicate that they use vim-slime and tmux.
    \item The vim-tmux-runner plugin \cite{toomey2013tmuxrunner} may be worth checking out. 
    \item The vim-ipython-cell plugin \cite{hanschen2019ipython} may be worth
        checking out. There is a reddit post \cite{hanschen2019reddit} by the
        author on this plugin. Note that this plugin is leverages vim-slime.
        According to the author, the main contribution of this plugin is that it
        provides many ways to define and run cells, even using Vim marks.
    \item Another lightweight workflow can be found in
        \cite{hornung2019boosting}. For this, you will need three plugins: Vimux
        \cite{benmills2009vimux}, vim-pyShell \cite{hornung2019pyShell} and
        vim-cellmode \cite{julienr2016vimcellmode}. However, I have trouble
        getting this to work. Starting and stopping a pyShell session works fine
        as well as sending one line of code, but I have issues sending over
        cells.
\end{itemize}


%---------------------------------------------------------------
%                         Bibliography
%---------------------------------------------------------------
\nocite{*}
\bibliography{references}
\bibliographystyle{abbrv}
\end{document}
