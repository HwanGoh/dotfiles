% SIAM Shared Information Template
% This is information that is shared between the main document and any
% supplement. If no supplement is required, then this information can
% be included directly in the main document.


% Packages and macros go here
\usepackage{lipsum}
\usepackage{amsfonts,dsfont}
\usepackage{graphicx}
\usepackage{epstopdf}
\usepackage{algorithmic}
\ifpdf
  \DeclareGraphicsExtensions{.eps,.pdf,.png,.jpg}
\else
  \DeclareGraphicsExtensions{.eps}
\fi

% Add a serial/Oxford comma by default.
\newcommand{\creflastconjunction}{, and~}

% Used for creating new theorem and remark environments
\newsiamremark{remark}{Remark}
\newsiamremark{hypothesis}{Hypothesis}
\crefname{hypothesis}{Hypothesis}{Hypotheses}
\newsiamthm{claim}{Claim}

% Sets running headers as well as PDF title and authors
\headers{<++>}{Others and Tan Bui-Thanh}

% Title. If the supplement option is on, then "Supplementary Material"
% is automatically inserted before the title.
\title{<++>\thanks{Submitted to the editors DATE.
\funding{This work was funded by some funds}}}

% Authors: full names plus addresses.
\author{Others
\and Tan Bui-Thanh\thanks{Department of Aerospace Engineering and
  Engineering Mechanics, The Oden Institute for Computational
  Engineering and Sciences, UT Austin,  Austin, Texas
  (\email{tanbui@ices.utexas.edu},
  \url{https://users.oden.utexas.edu/\string~tanbui/}).}
}
\usepackage{amsopn}
\DeclareMathOperator{\diag}{diag}


%% Added on Overleaf: enabling xr
\makeatletter
\newcommand*{\addFileDependency}[1]{% argument=file name and extension
  \typeout{(#1)}% latexmk will find this if $recorder=0 (however, in that case, it will ignore #1 if it is a .aux or .pdf file etc and it exists! if it doesn't exist, it will appear in the list of dependents regardless)
  \@addtofilelist{#1}% if you want it to appear in \listfiles, not really necessary and latexmk doesn't use this
  \IfFileExists{#1}{}{\typeout{No file #1.}}% latexmk will find this message if #1 doesn't exist (yet)
}
\makeatother

\newcommand*{\myexternaldocument}[1]{%
    \externaldocument{#1}%
    \addFileDependency{#1.tex}%
    \addFileDependency{#1.aux}%
}

\newcommand{\mynote}[3]{
	\textcolor{#2}{\fbox{\bfseries\sffamily\scriptsize#1}}
		{\textsf{\emph{#3}}}
}

\newcommand{\tanbui}[1]{\mynote{Tan}{magenta}{#1}}
%%% END HELPER CODE
%%% Local Variables: 
%%% mode:latex
%%% TeX-master: "ex_article"
%%% End: 

% Tan macros
\newcommand{\Grad} {\ensuremath{\nabla}}  % Gradient
\newcommand{\Div} {\ensuremath{\nabla\cdot}} % Divergence

\newcommand{\nor}[1]{\left\| #1 \right\|} % norm
\newcommand{\snor}[1]{\left| #1 \right|} %semi-norm
\newcommand{\LRp}[1]{\left( #1 \right)} % adaptive left and right parentheses
\newcommand{\LRs}[1]{\left[ #1 \right]} % adaptive left and right square brackets
\newcommand{\LRa}[1]{\left< #1 \right>} % adaptive left and right arrow brackets
\newcommand{\LRc}[1]{\left\{ #1 \right\}} % adaptive left and right curly brackets
\newcommand{\pp}[2]{\frac{\partial #1}{\partial #2}} % adaptive partial derivatives

\newcommand{\mc}[1]{\mathcal{#1}} %mathcal
\newcommand{\mb}[1]{\mathbf{#1}} %math boldface
\newcommand{\mbb}[1]{\mathbb{#1}} %mathbb
\newcommand{\half}{\frac{1}{2}}
\newcommand{\halfv}[1]{\frac{#1}{2}}


\renewcommand{\H}{H}
\newcommand{\C}{C}
\newcommand{\W}{W}
\newcommand{\We}{\W_{\mathrm{e}}}
\newcommand{\Wd}{W_{\mathrm{d}}}
\renewcommand{\L}{L}

\newcommand{\DNN}{\Psi}
\newcommand{\F}{\mc{G}}

\newcommand{\bs}[1]{\boldsymbol{#1}}
\renewcommand{\P}{U}
\newcommand{\U}{\P}
\newcommand{\Pbar}{\overline{\P}}
\newcommand{\Ubar}{\Pbar}
\newcommand{\Psie}{\Psi_{\mathrm{e}}}
\newcommand{\Psid}{\Psi_{\mathrm{d}}}
\newcommand{\bd}{{\bf b}_{\mathrm{d}}}
\newcommand{\be}{{\bf b}_{\mathrm{e}}}
\newcommand{\Bd}{B_{\mathrm{d}}}
\newcommand{\Be}{{B}_{\mathrm{e}}}
\newcommand{\bb}{{\bf b}}
\newcommand{\B}{B}
\newcommand{\Z}{Z}
\newcommand{\Y}{Y}
\newcommand{\Ybar}{\overline{\Y}}
\newcommand{\yb}{\bs{y}}
\newcommand{\ybbar}{\overline{\yb}}
\newcommand{\ybobs}{\bs{y}^{obs}}
\newcommand{\One}{\mathds{1}}
\newcommand{\nt}{{n_{\mathrm{t}}}}
\newcommand{\n}{n}
\newcommand{\m}{m}
\newcommand{\R}{{\mathbb{R}}}
\newcommand{\I}{I}
\newcommand{\J}{J}
\newcommand{\G}{G}
\newcommand{\zb}{\bs{z}}
\newcommand{\pb}{\bs{u}}
\newcommand{\ub}{\pb}
\newcommand{\pbbar}{\overline{\pb}}
\newcommand{\ubbar}{\pbbar}
\newcommand{\ubstar}{\ub^*}


\newcommand{\figlab}[1]{\label{fig:#1}}
\newcommand{\eqnlab}[1]{\label{eq:#1}}
\newcommand{\theolab}[1]{\label{theo:#1}}
\newcommand{\corolab}[1]{\label{coro:#1}}
\newcommand{\propolab}[1]{\label{propo:#1}}
\newcommand{\lemlab}[1]{\label{lem:#1}}
\newcommand{\defilab}[1]{\label{defi:#1}}
\newcommand{\remalab}[1]{\label{rema:#1}}
\newcommand{\tablab}[1]{\label{tab:#1}}

\newcommand{\figref}[1]{\ref{fig:#1}}
\newcommand{\theoref}[1]{\ref{theo:#1}}
\newcommand{\defiref}[1]{\ref{defi:#1}}
\newcommand{\remaref}[1]{\ref{rema:#1}}
\newcommand{\cororef}[1]{\ref{coro:#1}}
\newcommand{\proporef}[1]{\ref{propo:#1}}
\newcommand{\lemref}[1]{\ref{lem:#1}}
\newcommand{\eqnref}[1]{\eqref{eq:#1}}
\newcommand{\alglab}[1]{\label{alg:#1}}
\newcommand{\algref}[1]{\ref{alg:#1}}
\newcommand{\seclab}[1]{\label{sect:#1}}
\newcommand{\secref}[1]{\ref{sect:#1}}
\newcommand{\tabref}[1]{\ref{tab:#1}}
